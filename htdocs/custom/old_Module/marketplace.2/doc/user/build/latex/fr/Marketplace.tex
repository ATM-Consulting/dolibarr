%% Generated by Sphinx.
\def\sphinxdocclass{report}
\documentclass[letterpaper,10pt,english]{sphinxmanual}
\ifdefined\pdfpxdimen
   \let\sphinxpxdimen\pdfpxdimen\else\newdimen\sphinxpxdimen
\fi \sphinxpxdimen=49336sp\relax

\usepackage[margin=1in,marginparwidth=0.5in]{geometry}
\usepackage[utf8]{inputenc}
\ifdefined\DeclareUnicodeCharacter
  \DeclareUnicodeCharacter{00A0}{\nobreakspace}
\fi
\usepackage{cmap}
\usepackage[T1]{fontenc}
\usepackage{amsmath,amssymb,amstext}
\usepackage{babel}
\usepackage{times}
\usepackage[Bjarne]{fncychap}
\usepackage{longtable}
\usepackage{sphinx}

\usepackage{multirow}
\usepackage{eqparbox}

% Include hyperref last.
\usepackage{hyperref}
% Fix anchor placement for figures with captions.
\usepackage{hypcap}% it must be loaded after hyperref.
% Set up styles of URL: it should be placed after hyperref.
\urlstyle{same}
\addto\captionsenglish{\renewcommand{\contentsname}{Contents:}}

\addto\captionsenglish{\renewcommand{\figurename}{Fig.\@ }}
\addto\captionsenglish{\renewcommand{\tablename}{Table }}
\addto\captionsenglish{\renewcommand{\literalblockname}{Listing }}

\addto\extrasenglish{\def\pageautorefname{page}}

\setcounter{tocdepth}{1}



\title{Marketplace Documentation}
\date{May 30, 2019}
\release{development}
\author{Librethic}
\newcommand{\sphinxlogo}{}
\renewcommand{\releasename}{Release}
\makeindex

\begin{document}

\maketitle
\sphinxtableofcontents
\phantomsection\label{\detokenize{index::doc}}



\chapter{Introduction}
\label{\detokenize{intro:introduction}}\label{\detokenize{intro::doc}}\label{\detokenize{intro:welcome-to-marketplace-s-documentation}}
« Marketplace » est un module complémentaire pour Dolibarr ERP/CRM qui permet de gérer des ventes à rétrocession.


\section{Pré-requis}
\label{\detokenize{intro:pre-requis}}\begin{itemize}
\item {} 
Dolibarr version 8 au minimum ;

\item {} 
PHP dans sa version 5.6 (recommandée 7.+).

\end{itemize}
\href{https://dolistore.com}{\sphinxincludegraphics{{dolistore-logo}.jpg}}

\section{Fonctionnalités}
\label{\detokenize{intro:fonctionnalites}}
Le module apporte plusieurs fonctionnalités à Dolibarr ERP/CRM :
\begin{itemize}
\item {} 
la définition de taux de prélèvement, chaque taux étant lié à une catégorie fournisseur

\item {} 
ajout d'une liaison produit -\textgreater{} tier (fournisseur)

\item {} 
calcul du montant du reversement et sauvegarde lors de chaque vente

\item {} 
génération des factures de rétrocessions sur une période donnée

\end{itemize}


\section{À propos}
\label{\detokenize{intro:a-propos}}
Ce module est édité par \href{https://librethic.io/}{Librethic}. N'hésitez pas à nous soumettre toute anomalie ou idée à propos du module.


\section{Contribuer}
\label{\detokenize{intro:contribuer}}
Chacun/chacune peut contribuer au développement du module en achetant l'archive sur la place de marché des modules complémentaires à Dolibarr ERP/CRM : Dolistore (\href{https://www.dolistore.com/en/modules/xxx.html}{page du module Marketplace}).

Pour les développeurs, un espace projet est disponible sur la \href{https://code.librethic.io}{forge Librethic}. Il suffit de créer un compte afin de pouvoir utiliser le gestionnaire d'anomalie et/ou publier du code au sein du module.

Les traducteurs peuvent utiliser la plateforme Transifex pour la traduction du module.


\chapter{Usage}
\label{\detokenize{usage:usage}}\label{\detokenize{usage::doc}}

\section{Configuration}
\label{\detokenize{usage:configuration}}
Un écran de configuration permet de paramétrer le module :
\begin{quote}
\begin{description}
\item[{Catégorie fournisseur principale pour les taux de prélèvement}] \leavevmode
Cette catégorie fournisseur sera utilisée pour trouver les taux de prélèvement/commission.

\item[{Service utilisé pour les commissions}] \leavevmode
Indiquer un service qui sera utilisé pour les factures « pour le compte de tier »

\end{description}
\end{quote}


\section{Gestion des taux de prélèvement}
\label{\detokenize{usage:gestion-des-taux-de-prelevement}}
Des taux de prélèvement différents peuvent être paramétrés. Avant la création de taux de prélèvement il est nécessaire de crééer les catégories fournisseurs auxquelles on associera ensuite le taux de prélèvement.

\noindent{\hspace*{\fill}\sphinxincludegraphics{{screenshot-fr-liste-taux}.png}\hspace*{\fill}}


\section{Liaison produit / tiers (fournisseur)}
\label{\detokenize{usage:liaison-produit-tiers-fournisseur}}
Pour savoir qui vend quoi, il faut lier le ou les produits à un tiers qui doit être marqué comme fournisseur.

Le module ajoute la possibilité de saisir le revendeur (champ Vendeur rétrocession) lors de la création ou l'édition de chaque produit.

\noindent{\hspace*{\fill}\sphinxincludegraphics{{screenshot-fr-produit-fournisseur}.png}\hspace*{\fill}}

Pour que le calcul de commission ait lieu lors des ventes \sphinxstylestrong{il faut que le vendeur (fournisseur dans Dolibarr) soit classé dans une catégorie qui correspond à un taux de prélèvement}.


\section{Les commissions / prélèvements}
\label{\detokenize{usage:les-commissions-prelevements}}
Lors de chaque vente faite dans Dolibarr ERP/CRM, le module MarketPlace va procéder au calcul du montant de la commission pour chaque ligne de facture, en fonction du tiers lié au produit (le vendeur).
Le taux de commission est celui associé à la catégorie fournisseur à laquelle est lié ce tiers.

Plus de détails dans la section {\hyperref[\detokenize{ventes:rst-sales}]{\sphinxcrossref{\DUrole{std,std-ref}{Gestion des ventes}}}}.


\chapter{Gestion des ventes}
\label{\detokenize{ventes:gestion-des-ventes}}\label{\detokenize{ventes::doc}}\label{\detokenize{ventes:rst-sales}}

\section{Enregistrement des ventes / reversement}
\label{\detokenize{ventes:enregistrement-des-ventes-reversement}}
Chaque vente est enregistrée après paiement d'une facture, grâce aux triggers de Dolibarr ERP/CRM : on calcule le montant de reversement en trouvant le vendeur du produit et son taux de prélèvement.

La formule pour le calcul du montant à reverser est la suivante :

\begin{sphinxVerbatim}[commandchars=\\\{\}]
prix de base HT = (prix produit * quantité)
prix de base HT  \PYGZhy{} ( prix de base HT * ( taux prélèvement / 100)).
\end{sphinxVerbatim}

La formule pour le calul du montant à prélever est la suivante :

\begin{sphinxVerbatim}[commandchars=\\\{\}]
total HT de la ligne \PYGZhy{} montant à reverser (\PYGZhy{} montant pris en charge vendeur)
\end{sphinxVerbatim}


\section{Liste des ventes}
\label{\detokenize{ventes:liste-des-ventes}}
Pour chaque facture payée dans Dolibarr, le montant de la vente, le taux de rétrocession et le montant à rétrocéder sont listés sur l'écran accessible depuis le menu \sphinxstyleemphasis{Facturation/paiement -\textgreater{} Liste des ventes fournisseurs}.

\noindent{\hspace*{\fill}\sphinxincludegraphics{{marketplace-sales-list}.png}\hspace*{\fill}}

Pour chaque ligne de facture cette liste affiche :
\begin{quote}
\begin{description}
\item[{Vendeur}] \leavevmode
Le tiers (fournisseur) qui a fait la vente

\item[{Produit}] \leavevmode
Le produit vendu

\item[{Prix}] \leavevmode
Le prix de vente

\item[{Taux de prise en charge}] \leavevmode
…

\item[{Montant pris en charge}] \leavevmode
Montant pris en charge par le vendeur

\item[{Montant remise}] \leavevmode
Remise en pourcentage, effectuée sur la ligne

\item[{Taux de prélèvement}] \leavevmode
Le taux de prélèvement appliqué à la vente (en pourcentage)

\item[{Montant prélèvement}] \leavevmode
Montant du prélèvement

\item[{Taux taxe}] \leavevmode
Le taux de taxe appliqué à la ligne

\item[{Montant rétro-cession}] \leavevmode
Montant à verser au vendeur, après déduction de la commission

\item[{Facture client}] \leavevmode
Facture correspondante au client final

\item[{Facture vendeur}] \leavevmode
Facture du vendeur  (un tiers fournisseur pour Dolibarr)

\item[{État}] \leavevmode
Statut de la vente : non comptée, facturée ou payée

\end{description}
\end{quote}


\section{Facturation des ventes}
\label{\detokenize{ventes:facturation-des-ventes}}
Les montants des ventes à rétrocéder sont facturés à l'aide des factures fournisseurs de Dolibarr ERP/CRM.

Le module fourni un modéle spécifique pour la \sphinxstylestrong{facturation « pour le compte de tiers »}.

Un formulaire, accessible depuis le menu \sphinxstyleemphasis{Facturation / paiement -\textgreater{} Rétrocession -\textgreater{} facturer les ventes fournisseur}, permet de sélectionner la période des ventes à rétrocéder et éventuellement filtrer par un fournisseur.

\noindent{\hspace*{\fill}\sphinxincludegraphics{{screenshot-fr-form-invoices-generation}.png}\hspace*{\fill}}

Après un clic sur le bouton \sphinxstyleemphasis{Lancer la génération des factures} va déclencher le traitement suivant :
\begin{itemize}
\item {} 
regroupement des ventes à facturer par fournisseur dans la période sélectionnée

\item {} 
pour chaque fournisseur :
\begin{itemize}
\item {} 
création d'une facture avec décompte de chiffres d'affaires et une ligne négative pour les commissions.

\item {} 
changement du statut de la vente : de \sphinxstylestrong{À facturer} à \sphinxstylestrong{Attente règlement}

\item {} 
validation de la facture

\item {} 
envoi de la facture par mail (à venir)

\end{itemize}

\end{itemize}


\section{Suivi des paiements}
\label{\detokenize{ventes:suivi-des-paiements}}
Lorsqu'un paiement est émis il doit être saisi sur la facture fournisseur.

Une fois que la facture fournisseur est marquée comme payée, la vente passe du statut \sphinxstylestrong{Attente réglement} au statut \sphinxstylestrong{Payé}.


\section{Modification d'une vente}
\label{\detokenize{ventes:modification-d-une-vente}}
Chaque vente peut être modifiée en cliquant sur l'icône «stylo» en bout de chaque ligne de la liste des ventes.

\sphinxstylestrong{Attention} : l'édition des valeurs ne provoquera pas le recalcul des commissions, il faut saisir les bonnes données à calculer au préalable.


\chapter{Permissions}
\label{\detokenize{permissions::doc}}\label{\detokenize{permissions:permissions}}
Permissions pour accéder aux différents objets du module MarketPlace.
\begin{quote}
\begin{description}
\item[{marketplace-\textgreater{}read}] \leavevmode
Accéder au menu marketplace

\item[{marketplace-\textgreater{}tx-\textgreater{}read}] \leavevmode
Lire les taux de prélèvement

\item[{marketplace-\textgreater{}tx-\textgreater{}write}] \leavevmode
Ajouter / modifier les taux de prélèvement

\item[{marketplace-\textgreater{}tx-\textgreater{}delete}] \leavevmode
Supprimer les taux de prélèvement

\item[{marketplace-\textgreater{}sales-\textgreater{}read}] \leavevmode
Voir les ventes marketplace

\item[{marketplace-\textgreater{}sales-\textgreater{}export}] \leavevmode
Exporter les ventes : générer les factures de rétrocessions

\item[{marketplace-\textgreater{}sales-\textgreater{}write}] \leavevmode
Modifier les ventes marketplace (!!!)

\end{description}
\end{quote}

\noindent{\hspace*{\fill}\sphinxincludegraphics{{screenshot-marketplace-rights}.png}\hspace*{\fill}}



\renewcommand{\indexname}{Index}
\printindex
\end{document}